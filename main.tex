\documentclass[10pt]{article}

% Packages for formatting
\usepackage[margin=1in]{geometry}
\usepackage{setspace}
\usepackage{titlesec}
\usepackage{fancyhdr}

% % Header setup
% \fancypagestyle{firstpageheader}{%
% \fancyhf{}
% \renewcommand{\headrulewidth}{0pt} % Remove header line
% \fancyhead[C]{%
%     \textbf{\LARGE Academic Statement of Purpose} \\[1em] % Increased spacing between title and the next line
%     \textbf{\large Your Name} \\[0.5em] % Spacing between lines
%     \textbf{\large Graduate Program Name}
% }
% }

% % Adjust header-body spacing
% \setlength{\headsep}{70pt} % Increase space between header and body

% % Title formatting
% \titleformat{\section}{\large\bfseries}{\thesection.}{1em}{}

% Document begins
\begin{document}

% Title and Subtitles
\begin{center}
    {\LARGE \textbf{Statement of Purpose}} \\[1em] % Main title with spacing
    {\large \textbf{Ashwath Ram Suryanarayanan}} \\[1em] % Subtitle: Your name
    {\normalsize \textbf{PhD Applicant, Mechanical Engineering}} \\[1em] % Subtitle: Program name with extra spacing below
\end{center}

\noindent I had an early fascination in my life for vehicles and mechanical systems. Out of that grew my passion for robotics and since then, my journey has been shaped by a blend of curiosity, determination, and the pursuit of challenging goals. \\ \\ Throughout my academic journey, I have been drawn to understanding and solving complex problems in mechanical systems and robotics. My undergraduate and graduate research experiences have equipped me with a strong foundation in dynamics and control theory applied in robotics, and I am eager to delve deeper into these areas through a PhD at University of California, San Diego. I aim to contribute to the development of robust control strategies for dynamic and underactuated systems, with a particular focus on developing adaptive control systems for complex mechanical systems such as legged robots. \\ \\
\textbf{Undergraduate Foundations}\\ 
My early exposure to research came during an internship under Prof. Chandramouli Padmanabhan at IIT Madras, where I worked on developing a dynamic simulation model for a cantilever beam after introducing periodic variations in its material and geometry. By modelling the structural dynamics of the system using Finite element method (FEM) in MATLAB, I derived time-domain responses for the beam and validated the results of my baseline model with commercial tools like ANSYS before proceeding further with my simulations. I further explored the modal behaviour of the beams by conducting FFT analysis on displacement data to identify dominant frequencies. This was done to induce approximate geometrical and material variations such as attaching well-tuned spring mass dampers to isolate particular dominant modes of vibration through passive control. Although the internship was remote, this experience taught me the importance of computational modelling and numerical methods, providing a solid foundation for my subsequent research projects. \\ \\During my undergraduate studies in production engineering at NIT Trichy, I applied my learning in practical settings as part of the suspension and steering design team for PSI Racing, the institute’s motorsports club. My responsibilities included determining key parameters shaping the vehicle dynamics of the ATV, optimally designing components to meet these requirements, and ensuring structural rigidity. Collaborating closely with the Data Acquisition (DAQ) team, I worked on tuning and improving our subsystem based on the vehicle’s on-road performance, directly applying concepts from my internship under Prof. Chandramouli Padmanabhan. As we were using a pneumatic shock absorber, we developed a method to correlate air pressure with its stiffness and damping ratio. We built a DAQ system with a MEMS-based IMU to record acceleration data from the rear-left unsprung mass as it traversed bumps. By processing this data through FFT analysis, we identified the dominant frequency and the corresponding stiffness and estimated the damping ratio from the time-response decay. Being involved in this club enhanced my ability to translate theoretical knowledge into practical engineering solutions. It was also during this time that I began to see the broader applications of my skills, sparking an interest in exploring systems beyond automotive design. These experiences expanded my problem-solving skills and solidified my desire to explore robotics, where mechanics, control systems, and design converge. \\ \\
\textbf{Transition to Robotics: Research on Stewart Platform }\\
My transition to robotics began with my final-year project at IIT Delhi, where the study of the flexural behaviour of a Stewart Platform Manipulator (SPM), a closed-loop parallel robot, formed the first chapter. This work focused on understanding its structural rigidity through theoretical modelling and experimental validation. My primary objective was to identify the dominant mode and natural frequency which contributed most to its vibration as this study would aid in designing an SPM with closely banded fundamental frequencies. I conducted a detailed modal analysis using ANSYS and developed a MATLAB-based dynamic model, which I validated against experimental data. To capture vibration responses, I designed a data acquisition system using a MEMS-based IMU connected to the mobile platform and Arduino Mega to collect the three-axis acceleration data. This data was processed and analyzed to understand the flexural behaviour of platform with different materials undergoing various trajectories using FFT to identify dominant vibration modes and frequencies. Challenges such as low-frequency noise in the IMU signals were resolved with a bandpass filter, tuned based on simulation results. This chapter deepened my understanding of signal processing, noise reduction and DAQ systems and culminated in a research paper presented at a leading conference\cite{air}. \\ \\ The second phase of this project involved studying the kinematic and geometric limits of the SPM. I analyzed its reachable translational and orientational workspace through three distinct methods: analytical modeling in MATLAB, geometrical modeling in Creo, and experimental validation using image processing with Aruco markers and OpenCV. Addressing practical challenges, such as the limited field of view of the camera and inconsistencies in orientation matrices, I redesigned the experimental setup and applied Axis-angle formulation after constructing a coordinate system to the top platform. I presented my research on workspace estimation in iNaCoMM-2023 conference with the paper being published in Springer\cite{inac}. This comprehensive approach to studying the SPM’s workspace allowed me to refine my skills in calibration techniques, concepts from image processing and workspace optimization, while further solidifying my interest in robotics. \\ \\
\textbf{Graduate Research: Control of Underactuated Systems }\\ 
Building on my undergraduate experiences, I pursued an MS(R) at IIT Delhi under the guidance of Prof. S.K. Saha and Dr. Rama Krishna K. This provided me an opportunity to deepen my expertise in control systems and robotics. My coursework included advanced topics in linear and nonlinear control theory, which I applied to my research on underactuated systems. My current work focuses on developing adaptive control strategies for a 3R double inverted pendulum, a challenging system often used as a model for bipedal locomotion.\\ \\
As a preliminary step, I developed a MATLAB simulation of a single inverted pendulum using the Recursive Newton-Euler algorithm and DeNOC matrices employing a dynamic RK4 algorithm for numerical integration. I divided the control process into swing-up and stabilization phases, with each phase employing tailored strategies. This was done to address the address the issue of zero dynamics in exactly linearized controllers. I implemented a hybrid control scheme that combined feedback linearization for swing-up and LQR for stabilization, successfully achieving rapid stabilization within five seconds based on the simulation. \\ \\
Extending this work to the double inverted pendulum, I identified gaps in the literature, particularly the lack of robust equilibrium switching control strategies that effectively track the states of active joints while maintaining stability. My research seeks to address these gaps by designing controllers capable of transitioning between equilibrium points in a multi-link system. To complement this, I am developing adaptive controllers that integrate reinforcement learning to handle external disturbances and system uncertainties. \\ \\
I am also exploring the concept of a generalized pendulum model, which abstracts the dynamics of complex underactuated systems and higher-order pendulums. My goal is to use this model as a framework for designing control strategies that are scalable and versatile across different applications. \\ \\
My experimental work involves constructing a hardware model of a double inverted pendulum with a detachable second link, allowing flexibility to test controllers for both single and double pendulum configurations. This setup will also facilitate the design of observer-based state estimation methods. By combining theoretical modelling with experimental validation, my work aims to create robust and practical solutions for underactuated systems, laying the groundwork for applications in legged robotics and beyond. \\ \\
\textbf{Future Study }\\
Reflecting on my journey, I realize how each step has prepared me to pursue advanced studies in robotics. My fascination with mechanical systems as a student evolved into a passion for tackling more complex challenges in dynamic environments. Along the way, I developed an interest in the field of control of legged robots during my graduate studies, as I recognized the parallels between underactuated systems and the dynamics of bipedal locomotion. Assisting a senior of mine in building an experimental setup to validate his algorithm used to stabilize a perturbed biped under different impulse and stance conditions. This inspired me to go through the current literature on stabilization control algorithms for legged robots and I was motivated to study the Linear Inverted Pendulum Model (LIPM), a common approach for modelling the walking gaits of bipeds. My literature review revealed significant research gaps, including the need for unified control strategies to manage transitions between dynamic gaits and the lack of adaptive methods for handling uneven terrains and external disturbances. \\ \\
Inspired by these challenges, I aim to extend my research to develop robust and adaptive controllers for legged robots. I wish to work on either of the following (1) Designing strategies for smooth transitions between walking, running, and jumping gaits, (2) Improving impact absorption and vertical foot force distribution to realize stable locomotion, (3) Enabling navigation across unstructured environments as offline techniques like ZMP gait synthesis fail in these situations.  
Apart from legged robots, I am also interested to (4) Explore stabilization techniques for underactuated systems with higher order of passivity. \\ \\
\textbf{Why University of Michigan?}\\
The University of California San Diego’s Mechanical Engineering program stands out for its interdisciplinary research, established labs known for their collaborative environment, and emphasis on solving real-world challenges. I am particularly drawn to \textbf{Prof. Nicholas Gravish}’s work on legged robot design, specifically on the Legged locomotion and dynamic running control for small scale robots on uneven surfaces. \textbf{Prof. Michael Tolley}’s work in the Bioinspired Robotics and Design Lab on soft robotics fascinates me. I am particularly intrigued by the Lab’s design of controller for the locomotion of soft-legged robots using pneumatic circuits. I am also interested to work in MURO Lab headed by \textbf{Prof. Sonia Martinez} and \textbf{Prof. Jorge Cortes} as the research projects carried out by the group are highly application oriented.  \\ \\
By pursuing a PhD at the University of California San Diego, I hope to build on my academic and research foundations, contribute to cutting-edge projects, and collaborate with experts in the field. My long-term vision is to develop unified control strategies that bridge theoretical advancements and practical applications. Whether enabling robots to navigate complex terrains, designing adaptive controllers for dynamic gaits, or exploring underactuated systems, I hope to contribute to the transformative impact of robotics in society. I am confident that this program will equip me with the tools and opportunities to make meaningful contributions to the robotics community while advancing my career as a researcher and innovator.   

\bibliographystyle{plain} % Choose a bibliography style
\bibliography{references.bib} % Name of the .bib file (without extension)

\vfill

\end{document}
